% !TeX spellcheck = fr_FR
\documentclass[a4paper,twoside,11pt]{book}
\usepackage[utf8]{inputenc}
\usepackage[T1]{fontenc}
\usepackage{makeidx}
\usepackage{nomencl}
\usepackage{amsthm}
\usepackage{amssymb}
\usepackage{mathrsfs}
\usepackage{amsmath}


\title{Intégration, Probabilités et Processus Aléatoires}
\author{Benoît Bourbié}

\newtheorem{definition}{Définition}[section]

\makeindex
\makenomenclature

\begin{document}
\maketitle

\part{Intégration}

\chapter{Espaces mesurés}

L'idée de départ de la théorie de la mesure est d'assigner un nombre réel positif (la mesure de cet sous-ensemble) à  chaque sous-ensemble d'un ensemble donné, de manière à satisfaire certaines propriétés naturelles d'additivité (la mesure d'une réunion disjointe doit être la somme des mesures).

\section{Ensembles mesurables}

\begin{definition}
Soit E un ensemble quelconque. Une tribu (ou $\sigma$-algèbre) sur E est une
famille $\mathcal{A}$ de parties de E telle que ($ \mathcal{A} \subset \mathcal{P}(E)$) :
\begin{itemize}
\item $E \in \mathcal{A}$
\item $A \in \mathcal{A} \Rightarrow A^{c} \in \mathcal{A}$
\item Si $A_{n} \in  \mathcal{A} $ pour tout $n \in \mathbb{N}$, on a a aussi $\bigcup\limits_{n \in \mathbb{N}} A_{n}  \in \mathcal{A}$.
\end{itemize}
\end{definition}

Les éléments de $\mathcal{A}$ sont appelés les parties mesurables de la tribu
$\mathcal{A}$, ou parfois $\mathcal{A}$-mesurables si il y a ambiguïté.


On dit que E est un espace mesurable.


Énonçons quelques conséquences de la définition:


\begin{itemize}
\item $\varnothing \in  \mathcal{A}$
\item Si $A_{n} \in  \mathcal{A} $ pour tout $n \in \mathbb{N}$, on a a aussi $\bigcap\limits_{n \in \mathbb{N}} A_{n}  \in \mathcal{A}$.
\item Puisque nous pouvons toujours prendre $A_n = \varnothing$ pour n assez grand, $\mathcal{A}$ est stable par unions (ou intersections) finis.
\item $\mathcal{A}$  est stable par intersections ou par unions finis ou dénombrables mais dans le cas indénombrables, ce n'est pas défini.
\end{itemize}

\paragraph*{Exemples:}
\begin{itemize}
\item $ \mathcal{A} = \mathcal{P}(E)$
\item $ \mathcal{A} = {  \varnothing , E} $
\item l'ensemble des parties de E qui sont (au plus) dénombrables ou dont le complémentaire
est (au plus) dénombrable forme une tribu sur E.
\end{itemize}


Pour avoir des exemples plus intéressants, on remarque qu'une intersection quelconque de tribus est encore une tribu. Ceci conduit à la définition suivante.

\begin{definition} Soit  $\mathcal{C}$ un sous-ensemble de $\mathcal{P}(E)$. Il existe alors une plus petite tribu sur E qui contienne $\mathcal{C}$. Cette tribu notée $\sigma(\mathcal{C})$ peut etre définie par

$$
	\sigma(\mathcal{C}) = \bigcap_{\mathcal{A} \text{tribu t.q.}  \mathcal{C} \subset \mathcal{A} } \mathcal{A}
$$

$\sigma(\mathcal{C})$ est appelée la tribu engendrée par $ \mathcal{C}$.
\end{definition}

\paragraph*{Tribu borélienne.} Pour donner un premier exemple de l'intérêt de la notion de tribu engendrée, considérons le cas où $E$
est un espace topologique.

\end{document}

